% Created 2024-07-02 Tue 09:12
% Intended LaTeX compiler: pdflatex
\documentclass[11pt]{article}
\usepackage[utf8]{inputenc}
\usepackage[T1]{fontenc}
\usepackage{graphicx}
\usepackage{longtable}
\usepackage{wrapfig}
\usepackage{rotating}
\usepackage[normalem]{ulem}
\usepackage{amsmath}
\usepackage{amssymb}
\usepackage{capt-of}
\usepackage{hyperref}
\author{Lucas Barros de Assis}
\date{}
\title{Analisando o impacto dos processadores na programação com aceleradores}
\hypersetup{
 pdfauthor={Lucas Barros de Assis},
 pdftitle={Analisando o impacto dos processadores na programação com aceleradores},
 pdfkeywords={},
 pdfsubject={},
 pdfcreator={Emacs 28.2 (Org mode 9.5.5)}, 
 pdflang={English}}
\begin{document}

\maketitle

\section{Contexto do trabalho e motivação}
\label{sec:org59c17aa}

Os computadores tupi do PCAD estão equipados com aceleradores NVIDIA RTX 4090, porém com processadores diferentes.
Diante do paralelismo dos aceleradores, muito superior ao dos processadores de uso geral, é possível questionar até que ponto a
capacidade dos processadores influencia no desempenho dessas máquinas.

A partir disso, queremos responder qual o impacto do desempenho dos processadores na programação com aceleradores.

\section{Metodologia}
\label{sec:org2ebf95f}

Para fins de comparação, a biblioteca de álgebra linear densa Chameleon será utilizada para realizar a fatoração de Cholesky.
Inicialmente serão criados alguns conjuntos de matrizes, cada um composto por matrizes de um dado tamanho, que serão utilizadas nos experimentos.
Em seguida, uma aplicação será desenvolvida para realizar a fatoração da biblioteca Chameleon em conjunto com o suporte de
execução StarPU, através do qual será possível definir um balanceamento de carga de maneira que apenas os aceleradores
realizem os cálculos propriamente ditos, deixando os processadores responsáveis apenas pelo escalonamento das tarefas.

A partir das matrizes previamente criadas, diversas execuções com matrizes de tamanhos e partições diferentes serão realizadas em cada um dos computadores,
registrando o seu tempo total de execução, o tempo de cálculo nos aceleradores e também o seu rastro de execução.

\section{Resultados esperados}
\label{sec:org64397c2}

Intuitivamente é esperado que os tempos de execução totais sejam ligeiramente diferentes, já que a transferência dos dados para os aceleradores deve ser
afetada pela sua tecnologia. No entanto, o tempo de cálculo nos aceleradores deve ser muito similar, já que tratam-se dos mesmos equipamentos realizando
os mesmos cálculos. Os traços de execução também não devem apresentar grandes diferenças entre os dois processadores já que, independente do tamanho
da tarefa, a quantidade de trabalho desempenhada pelos aceleradores deve ser a mesma.

\section{Cronograma}
\label{sec:orgd6ca91e}

O cronograma da elaboração desse trabalho é apresentado na Tabela \ref{tab:orgbac04de}.

\begin{table}[htbp]
\caption{\label{tab:orgbac04de}Cronograma de atividades.}
\centering
\begin{tabular}{ll}
Atividade & Quinzena\\
\hline
Desenvolvimento da aplicação para criação e armazenamento de matrizes & Junho/2\\
Desenvolvimento da aplicação da fatoração usando StarPU & Junho/2\\
Desenvolvimento dos scripts para realização dos experimentos & Julho/1\\
Execução dos experimentos & Julho/2\\
Análise dos resultados & Agosto/1\\
Elaboração do relatório e apresentação & Agosto/1\\
\hline
\end{tabular}
\end{table}


\section{Ambiente de programação}
\label{sec:org411a596}

Os experimentos serão realizados no PCAD utilizando as máquinas tupi.
\end{document}